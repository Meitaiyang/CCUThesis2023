\chapter{Introduction}
\label{c:intro}


自動中英文對照\gRef{SLAM}。沒有縮寫用\gRefNA{monocular}
Attention plays an important role in human vision. For example, when
we look at an image, our eye movements comprise a succession of {\em
fixations} (repetitive positioning of eyes to parts of the image)
and {\em saccades} (rapid eye jump). Those parts of the image that
cause eye fixations and capture primary attention are called {\em
regions of interest} (ROIs). Studies in visual attention and eye
movement have shown that humans generally only attend to a few ROIs.
Detecting these visually attentive regions in images is challenging
but useful in many multimedia applications, such as automatic
thumbnail cropping, object recognition, content-based image
retrieval, adaptive image compression and automatic browsing in
small-screen devices.\cite{vinet1989universal}

\begin{equation} \label{eq:01}
\begin{split}
A & = \frac{\pi r^2}{2} \\
& = \frac{1}{2} \pi r^2
\end{split}
\end{equation}

Many algorithms have 公式\ref{eq:01} been proposed for automatic ROI detection in
images. Unfortunately, these methods were often evaluated only on
specific and small data sets that are not publicly available. The
lack of published {\em benchmarks} makes experiments non-repeatable
and quantitative evaluation difficult. However, as recommended by
the latest ACM SIGMM retreat, repeatable experiments using published
benchmarks are important for advancing the multimedia research
field~\cite{Rowe:2005:ASR}.

\begin{figure}
	\centering
	\includegraphics[width=0.45\textwidth]{kl}
	\caption{kl-distance\cite{Rowe:2005:ASR}}
	\label{kl}
\end{figure}

\begin{table}[t]
\begin{center}
\begin{tabular}{lcc}

\hline
                    &  {\small Itti's method}     & {\small Fuzzy growing}    \\
\hline
{\small Precision}           &  0.4475    & 0.4506 \\
{\small Recall}              &  0.5515    & 0.5542 \\
\hline

\end{tabular}
\caption[Evaluation of FOA sets]{\small Evaluation of FOA sets. } \label{t:FOA}
\end{center}
\end{table}


%------------------------------------------------------------------------------
\begin{figure}[p]
	\begin{minipage}{0.5\textwidth}  %% {0.18\textwidth}
		\centerline{\includegraphics[width=1\textwidth]{kl}}
		\centerline{(a)01}
	\end{minipage}
	\hfill
	\begin{minipage}{0.5\textwidth}  %% {0.18\textwidth}
		\centerline{\includegraphics[width=1\textwidth]{kl}}
		\centerline{(c)01}
	\end{minipage}
	\vfill
	\begin{minipage}{0.5\textwidth}  %% {0.18\textwidth}
		\centerline{\includegraphics[width=1\textwidth]{kl}}
		\centerline{(b)04}
	\end{minipage}
	\hfill
	\begin{minipage}{0.5\textwidth}  %% {0.18\textwidth}
		\centerline{\includegraphics[width=1\textwidth]{kl}}
		\centerline{(d)04}
	\end{minipage}
	\captionsetup{justification=centering}
	\caption[多張圖片]{多張圖片\\ (a)-(b)說明 \\ (c)-(d)說明文字}
	\label{fig:exp_own_acc}
\end{figure}
%------------------------------------------------------------------------------
