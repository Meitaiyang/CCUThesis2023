\begin{abstractzh}
許多線上學習資源現今公開在網路上,像是開方式課程(Open Coures Ware, OCW)或是大規模開放線上課堂(Massive Open Online Course, MOOC),學習者可能會碰到過度選擇(Overchoice)問題。本研究提供不採用使用者資料之標籤式推薦演算法來解決過度選擇問題。在本系統中,採用了Yake模型進行關鍵字生成以及DBPedia之本體論架構增加標籤之多樣性,針對國立中正大學之課程進行推薦,並利用標籤相似度分析推薦給使用者其他校外學習資源。在實驗中,本研究採用了AB測試來進行驗證並證明使用者對於本研究架構較其他架構有高度的接受性。

\bigbreak
\noindent \textbf{關鍵字:}{\, \makeatletter \@keywordszh \makeatother}
\end{abstractzh}

\begin{abstracten}
Many extracurricular resources have been published on the internet, like MOOCs and OCW, so the learner who is interested in specific knowledge will encounter the over-choice problem. This work presents a tag-based recommendation system without user profiles to approach over-choice problems. In our system, we use the YAKE model to generate keywords and the Dbpedia ontology database to increase the diversity of the keywords for courses offered by National Chung Cheng University. The system will recommend other courses and extracurricular MOOCs with similar tags to the learner. In the experiment, we use the AB testing method to show that our system can recommend the courses to learners with higher satisfaction than other methods.

\bigbreak
\noindent \textbf{Keywords:}{\, \makeatletter \@keywordsen \makeatother}
\end{abstracten}

\begin{comment}
\category{I2.10}{Computing Methodologies}{Artificial Intelligence --
Vision and Scene Understanding} \category{H5.3}{Information
Systems}{Information Interfaces and Presentation (HCI) -- Web-based
Interaction.}

\terms{Design, Human factors, Performance.}

\keywords{Region of interest, Visual attention model, Web-based
games, Benchmarks.}
\end{comment}
