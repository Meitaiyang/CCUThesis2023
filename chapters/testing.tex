\chapter{實驗}
\label{chapter:testing}

\section{實驗方法}
\label{sec:testing_method}

\begin{table}[ht!]
    
    \caption{實驗之混淆矩陣}
    \centering

    \resizebox{\columnwidth}{!}{\begin{tabular}{|c|c|c|c|}
        \hline
        \diagbox{True}{Predicted} & \begin{tabular}[c]{@{}c@{}}Open\\ Drawer 1\end{tabular} & \begin{tabular}[c]{@{}c@{}}Open\\ Drawer 2\end{tabular} & \begin{tabular}[c]{@{}c@{}}Open\\ Drawer 3\end{tabular} \\ \hline
        Open Drawer 1 & 62\% & 0\% & 0\% \\ \hline
        Open Drawer 2 & 28\% & 35\% & 0\% \\ \hline
        Open Drawer 3 & 1\% & 15\% & 63\% \\ \hline
    \end{tabular}}
    \label{table:Exp4ConfuseMatrix}
\end{table}

\section{實驗一}
\label{sec:testing_exp1}

\begin{descitemize}
    \item[實驗目的]
    
    我們將依照\ref{sec:testing_method}節所述的方法,進行實驗。
    
    \item[實驗步驟] 
    
    實驗方法如下:

\end{descitemize}