\chapter{緒論}
\label{chapter:introduction}

In recent years, many extracurricular resources have been published on the internet, like Massive Open Online Courses (MOOCs) and Open Course Ware (OCW), so learners can easily learn about any topic of interest. These learning resources that can be used at a computer without restrictions on the time and place and reduce the physical limitations while learning, and can be regarded as e-learning technology \cite{tan2008}. Many universities have provided e-learning resources. However, learners cannot quickly choose the resources to appropriate their requirements, since there are too many choices, which might decrease the learning effect. 
A tag-based recommendation system is one of the approaches to this problem. The recommendation system is an information-filtering technique that solves the over-choice problems \cite{zhang2019}. This system will analyze the properties of extracurricular resources, generate related tags, and recommend them to the learner by finding other resources with similar tags. There are some mainstream methods to implement recommendation systems: content-based filtering, collaborative filtering, and hybrid method \cite{nagarnaik2015}. The most common e-learning recommendation system is a collaborative-based filtering recommendation system, which focuses on customized recommendations based on user information \cite{ali2022}. To the best of our knowledge, there are no studies using tag-based, content-based recommendation systems without user profiles. 
In this study, we propose a tag-based recommendation system without user profiles that will recommend other courses and MOOCs provided by MIT, Coursera, and Udemy, related to courses offered by National Chung Cheng University.

